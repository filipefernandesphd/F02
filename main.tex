% ====================================
% Template NÃO oficial do formulário F-02 Plano Individual de Trabalho do IF Sudeste MG em LaTeX.
% Versão 1.0.1
% Desenvolvido e mantido por:
%   Filipe Fernandes, PhD
%   https://www.linkedin.com/in/filipefernandesphd
%   filipe.fernandes@ifsudestemg.edu.br
% ====================================
\documentclass{article}
\usepackage{template}

\begin{document}

\section*{\background{\centering F.02 PLANO INDIVIDUAL DE TRABALHO}}

\begin{tcolorbox}[colback=gray!20, boxrule=0pt, arc=0pt, left=5pt, right=5pt, top=5pt, bottom=5pt]
( ) BIC Jr. (PIBIC Jr./FAPEMIG; PIBIC-EM/CNPq; PIBIC Jr./IF SUDESTE MG)

( ) BIC (PIBIC/FAPEMIG; PIBIC/CNPq; PIBIC-Af/CNPq; PIBIC/IF SUDESTE MG)

( ) PIBITI/CNPq (PIBITI/IF SUDESTE MG)

( ) PARCERIA/CADASTRO NA PROPPI IF SUDESTE MG:

( ) OUTROS: 
\end{tcolorbox}

\section*{\background{PLANO DE TRABALHO}}

\subsubsection*{TÍTULO DO PROJETO:}

\subsubsection*{LOCAL DE REALIZAÇÃO DO PROJETO (DEPARTAMENTO, NÚCLEO, LABORATÓRIO, SALA, ETC.):}

\begin{table}[h]
    \centering
    \begin{tabular}{l l}
        \textbf{( ) GRADUAÇÃO}   &   \textbf{( ) ENSINO MÉDIO/TÉCNICO} \\
        \textbf{( ) BOLSISTA}    &   \textbf{( ) VOLUNTÁRIO} \\    
    \end{tabular}
\end{table}

\section*{\background{MOTIVAÇÃO E METODOLOGIA}}
% Envolvimento e inserção do(a) bolsista/voluntário no desenvolvimento da proposta e detalhamento das estratégias a serem utilizadas para atendimento das metas

\section*{\background{RESULTADOS ESPERADOS}}
% Destacar a possibilidade de desenvolvimento de produto ou processo que inclua patente ou publicação;
% Destacar também os principais resultados esperados em função do projeto proposto no que diz respeito ao engajamento na proposta e absorção dos(as) bolsistas/voluntário pelo mercado de trabalho, graduação, pós-graduação, empresas, etc.

\section*{\background{CRONOGRAMA DE ATIVIDADES}}
% Relacionar as atividades a serem desenvolvidas

\begin{enumerate}
    \item 
\end{enumerate}

\section*{\background{CRONOGRAMA DE TRABALHO}}

\begin{table}[!ht]
    \centering
    \renewcommand{\arraystretch}{1.2} % Define a altura padrão das linhas
    \begin{tabularx}{\textwidth}{
        |>{\arraybackslash\RaggedRight}p{5cm}
        |>{\centering\arraybackslash}X
        |>{\centering\arraybackslash}X
        |>{\centering\arraybackslash}X
        |>{\centering\arraybackslash}X
        |>{\centering\arraybackslash}X
        |>{\centering\arraybackslash}X
        |>{\centering\arraybackslash}X
        |>{\centering\arraybackslash}X
        |>{\centering\arraybackslash}X
        |>{\centering\arraybackslash}X
        |>{\centering\arraybackslash}X
        |>{\centering\arraybackslash}X|
    }
        \hline
        \multicolumn{4}{|l|}{\textbf{DATA INICIAL}: mês/ano }    & \multicolumn{9}{l|}{\textbf{DATA FINAL}: mês/ano} \\ \hline
        \multicolumn{13}{|l|}{\textbf{DURAÇÃO DA BOLSA}: X meses } \\ \hline
        \rowcolor{gray!20} &   \multicolumn{12}{c|}{\textbf{MÊS DE TRABALHO}}  \\ \cline{2-13} 
        \rowcolor{gray!20}\textbf{ATIVIDADE} & \textbf{1} & \textbf{2} & \textbf{3} & \textbf{4} & \textbf{5} & \textbf{6} & \textbf{7} & \textbf{8} & \textbf{9} & \textbf{10} & \textbf{11} & \textbf{12} \\ \hline
        Atividade 1                &  &  &  &  &  &  &  &  &  &  &  &  \\ \hline
    \end{tabularx}
    \caption{Cronograma de trabalho}
    \label{tab:cronograma}
\end{table}


\def\refname{\background{REFERÊNCIAS BIBLIOGRÁFICAS}}
\bibliography{referencias}
\end{document}
